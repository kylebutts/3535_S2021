\documentclass[11pt]{article}

\author{\normalsize Kyle Butts\\{\footnotesize Univ. of Colorado, Boulder}}
\date{\footnotesize\today}

% Include Theme
\input{preamble.tex}


\begin{document}

\begin{center}
    \color{navyblue}
    {\Large ECON 3535 Spring 2021 \\ \textit{Literature Review Assignment}}

    Due: April $4^{th}$
\end{center}

\noindent
Academic papers in economics are built around a research question. For instance, mine is ``How does regulation distort the incentives of natural gas plant owners to retire old plants?'' The goal of this assignment is for you to learn how to use published academic research in economics to begin to answer an interesting question. A good literature review will not necessarily give a definitive answer to your research question, but it will explain to what extent there is consensus, and perhaps where there is disagreement or more analysis needed.

\noindent
\textbf{Instructions}: 

\begin{enumerate}
    \item  Pick a research question related to natural resource economics . It does not have to be something we have covered in class. It should be broad enough to have several related papers, but not so broad that there are a hundred.
    a) Too broad: ``How does economic growth affect the environment?''
    b) Too narrow: ``How does the US shift towards an information-based economy affect environmental outcomes of its trade partners?''
    c) Just right: ``How well does the Environmental Kuznets curve describe reality?''

    \item  Show me your proposed research question early. This is not for a grade, but it is for your benefit. Submit this through the canvas assignment called ``Literature Review: research question proposal''.
    
    \item  Around 1,500 words. Be concise. Word count does not include references.
    
    \item  Sources. For this assignment, use at least 8 academic-quality sources that form the basis of your paper. Include a references section at the end. Any style works, but be consistent.
    a) Peer-reviewed research papers published in academic journals are ideal
    b) Formal papers or reports published by government agencies and well-respected research groups are also good.
    
    \item  Format. Please use a Word/Google Docs document saved as a pdf, double-spaced, with a normal font. Submit via Canvas. Deadlines are strict, so no special pleading.
\end{enumerate}


\noindent
\textbf{Grading guidelines}:

\begin{enumerate}
    \item  Analysis must be relevant- how well does your paper relate to the research question?
    
    \begin{enumerate}
        \item There’s a tendency to fill in the paper with unnecessary details. Some amount of exposition will be helpful, but keep in mind that your goal is not to explain everything about a topic. Your goal is to explain how the published research helps to answer your research question.
    \end{enumerate}
    
    \item  Sources must be relevant to your research question- does your use of sources help answer the question?
    
    \begin{enumerate}
        \item It’s important that the reader understands why you use a certain source. Sometimes it will be obvious. Other times the connection to the research question needs to be explained.
    \end{enumerate}

    \item  A good literature review will analyze the connections between sources- does your analysis
    build upon itself in a coherent way, or is it just a pile of semi-related thoughts?
    
    \begin{enumerate}
        \item Try to make connections, identify consensus, and explain disagreements. This is challenging. I won’t hold it against you if you can’t figure out why two papers disagree, but effort will be rewarded.
    \end{enumerate}
\end{enumerate}


\noindent
\textbf{Advice}:

\begin{enumerate}
    \item  Academic papers are annoying to read. For this assignment, you can just focus on the abstract
    and/or introduction (the only part of the paper anyone ever reads anyway).
    \item  One very useful way to find sources is to find a paper you like, and look at its citations (at the
    end of the paper) and the ``cited by \underline{\hspace{10mm}}'' option on Google Scholar
    a) Google Scholar is very helpful. If you search for a paper you want to cite, you will see a quotation mark symbol under each search result. Click that, and you get a pre-made citation to copy-paste for your bibliography section.

    \item  Here are a few of good places to look for research questions. You probably shouldn’t use them as actual sources, but they will help you find ideas.

        \begin{enumerate}
            \item www.greentechmedia.com

            \item Energyathaas.wordpress.com

            \item https://www.vox.com/energy-and-environment

            \item ``Example'' and ``Debate'' sections in the textbook
        \end{enumerate}

    \item To find sources: 

        \begin{itemize}
            \item Google Scholar $\to$ Advanced Search $\to$ Return articles published in \underline{\hspace{10mm}}
            
            \item This is a list of the top journals for environmental and resource economics. There are also ``general interest'' journals (like the American Economic Review) where you may find relevant papers.
            
            \item Energy Policy
            
            \item Journal of Environmental Economics and Management
            
            \item Energy Economics
            
            \item Renewable and Sustainable Energy Reviews
            
            \item Ecological Economics
            
            \item Applied Energy
            
            \item The Energy Journal
            
            \item Energy
            
            \item Environmental and Resource Economics
            
            \item Review of Environmental Economics and Policy
            
            \item Resource and Energy Economics
            
            \item Note: In order to access these journals, you will have to be on campus wifi or else Boulder VPN \href{https://oit.colorado.edu/services/network-internet-services/vpn}{https://oit.colorado.edu/services/network-internet-services/vpn}
        \end{itemize}
    
\end{enumerate}
    


\end{document}